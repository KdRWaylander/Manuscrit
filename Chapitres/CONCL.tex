\chapter*{Discussion générale et perspectives}
\addcontentsline{toc}{part}{Discussion générale et perspectives}
\markboth{DISCUSSION GÉNÉRALE ET PERSPECTIVES}{}

\lettrine[lines=3]{O}{N} conclue dans ce dernier chapitre l'ensemble des travaux relatifs à la thèse. On fait, dans un premier temps, la synthèse des résultats (modélisation et expérimentations) que l'on a obtenu, permettant ainsi d'avoir un modèle de score de réalisme. On applique ensuite ce dernier à différents systèmes présents chez Renault: casques de Réalité Virtuelle, casque de Réalité Augmentée, CAVEs, ... Pour terminer, on met en avant des applications concrètes qui pourraient utiliser notre modèle de score.

\section*{Synthèse des résultats}
	\subsection*{État des lieux de la modélisation}
	\par On a donc proposé un modèle de score de réalisme, basé sur le système visuel humain. Ce score ne se veut pas une nouvelle modélisation des fonctions visuelles humaines mais plutôt un guide pragmatique, pour l'ingénieur, pour estimer la qualité d'un système et sa capacité à envoyer des signaux de manière réaliste, c'est à dire conforme aux stimulation sensorielle dont on a l'habitude, indépendamment de ce qui est affiché.
	
	\par Ce score est divisé en douze critères (Fig. \ref{fig:score_realisme_mini}), répartis en deux entités de six: un groupe représentant les indices nécessaires à la vision, et un autre groupe représentant les indices d'immersion. Chaque critère est alors noté entre 0 et 100, basé sur les grandeurs qui caractérisent le système visuel humain. Une pondération des critères vient compléter le modèle et permet de prendre en compte l'utilisation que l'on fait du moyen immersif: en fonction de l'application, certains critères seront plus limitant que d'autres.
	
	\par Le groupe d'indices de vision se compose des critères suivants: contraste et luminance, nombre d'images par seconde, quotité de couleurs affichables, taille du champ de vision, acuités monoscopique et stéréoscopique. Parallèlement, les critères d'immersion sont les suivants: latence, taille du champ de regard, stéréoscopie, tracking, uniformité des surfaces d'affichage et convergence des caméras. Enfin, on propose une pondération par ordres de grandeur (fort ou faible) pour chaque critère.
	
	\begin{figure}
		\renewcommand\thefigure{I}
		\centering
		\includegraphics[scale=1]{Figures/ScoreRealismeMini}
		\caption{Rappel de la modélisation du score de réalisme.}
		\label{fig:score_realisme_mini}
	\end{figure}
	
	\subsection*{Résultats expérimentaux}
	\par Dans le cadre de la thèse, on a réalisé trois expérimentations: la validation d'un modèle de performance visuelle pour le critère de contraste et luminance, la mesure de performance pour le critère de latence et la comparaison entre les notations subjectives des critères du modèle par des sujets et les notes objectives données par notre modèle. Cette dernière, de moindre ampleur par rapport aux autres, dégage néanmoins quelques tendances. La pondération des critères de champ de vision et de champ de regard est très dépendante de l'application: dans le cas de la conduite, l'utilisation du champ de regard est quasi-nulle et les sujets ne font pas la différence entre les deux champs. L'appréciation de la quotité de couleur semble sur-évaluée par habitude. Enfin, les femmes font preuve de plus de sévérité dans leur notation que les hommes.
	
	\par Pour l'expérimentation sur le critère de contraste et de luminance, on a cherché à traduire une expérimentation en Réalité Virtuelle, afin de déterminer si le modèle de performance visuelle mis en jeu était utilisable dans notre domaine. Ce modèle prédit des conditions de luminance et de contraste en dessous desquelles le système visuel humain n'est pas capable de distinguer ce qu'on lui demande, et des conditions au dessus desquelles la performance de détection de l'œil n'augmente plus, déterminant ainsi la plage dans lequel un moyen immersif devrait se situer. Nos résultats théoriques semblent se comporter de manière radicalement différente par rapport aux prévisions du modèle, tout en montrant une corrélation statistique. On montre que le modèle de performance visuelle est utilisable, sous réserve de l'inclusion d'un facteur prenant en compte les spécificités de la Réalité Virtuelle. Pour l'application à notre critère de contraste et luminance, on revient toutefois à préconiser une technique utilisant les fonctions de sensibilité au contraste et la découpe en fréquence spatiale des tâches classique de vision.
	
	\par Enfin, l'expérimentation sur la latence est quant à elle une comparaison entre deux systèmes et à différents niveaux de latence dans chaque système, de la performance de sujets à réaliser une tâche écologique. Les sujets devaient viser une série de cibles dans un véhicule modélisé en 3D, en maximisant leur précision par rapport au centre de la cible et leur vitesse de réalisation. On montre notamment que la performance se dégrade de manière non continue sous l'influence de la latence, tandis que le mal du simulateur est lui linéairement affecté. De même, le changement de système (de CAVE à casque) améliore la performance des sujets mais au détriment de leur expérience utilisateur: on met en cause la nature des mouvements impliqués pour la réalisation de la tâche demandée ainsi que le conflit visio-vestibulaire.

	\subsection*{Travaux futurs}
	\par Si on présente une modélisation cohérente, il reste néanmoins des axes de travail et des propositions à améliorer ; le temps imparti pour la thèse ne nous ayant pas permis de tout traiter. Tout d'abord, bien que l'on ait eu des avancées concrètes pour le critère de contraste et luminance, nous ne sommes pas encore en mesure de pouvoir déterminer pratiquement son score. Par conséquent, le critère d'uniformité est encore à l'état d'embryon, étant intrinsèquement lié au contraste et à la luminance. On présente et on initie la suite des travaux sur le critère de contraste et de luminance avec l'utilisation des fréquences spatiales et des fonctions de contraste.
	
	\par D'autres critères cependant, tels que la quotité de couleur, la stéréoscopie et le tracking se sont vus attribuer une fonction de notation (linéaire ou binaire) que l'on juge insuffisante à terme ; ces notations sont fonctionnelles mais doivent être raffinées. On a présenté, par exemple dans le cas de la stéréoscopie, une piste d'amélioration de la fonction de notation, basée sur la technologie utilisée.
	
	\par La plus grande partie de futurs travaux reste la pondération du modèle. Si on propose des ordres de grandeur, il sera nécessaire, dans une second temps, d'avancer des propositions chiffrées, amenant plus de finesse pour départager les critères. Ces pondérations chiffrés seront très dépendantes du cas d'utilisation et permettront de prendre en compte certains scénarios spécifiques (comme la haute vitesse par exemple).

\section*{Application du score de réalisme}
\par Si la thèse a pour l'instant eu une application soit théorique, soit pratique mais au service de la théorie, il manque encore une application purement pratique de notre modèle de score. Or, on est désormais globalement en mesure d'appliquer notre modèle de score de réalisme visuel, à l'exception de deux critères, intrinsèquement liés, que sont le critère de contraste et luminance et le critère d'uniformité. On sélectionne des moyens immersifs présents chez Renault et on leur applique, sur chacun de leurs critères, les fonctions de notation déployées au cours de la thèse. Pour les critères non-aboutis, on utilise des estimations guidées par la littérature, à défaut de pouvoir les noter directement. De même, pour la pondération, on applique un coefficient uniforme de $2$ pour les critères <<~forts~>> et de $1$ pour les critères <<~faibles~>>. On sélectionne les systèmes suivants: deux CAVEs de chez Renault (nommés IRIS et P3I), un casque de Réalité Virtuelle grand public mais utilisé fréquemment dans l'industrie (HTC Vive) et un casque de Réalité Augmentée (Hololens). Techniquement, notre modèle de score n'est pas conçu pour une application en Réalité Augmentée, mais il peut être intéressant d'en tester les limites sur de tels dispositifs. Les résultats généraux de notation sont représentés en Fig. \ref{fig:resultats_score_application}.

\begin{figure}
	\renewcommand\thefigure{II}
	\centering
	\includegraphics[scale=1]{Figures/ResultatsScoreApplication}
	\caption{Score de réalisme pour les quatre systèmes testés.}
	\label{fig:resultats_score_application}
\end{figure}

\par IRIS est un dispositif de type CAVE et possède 5 faces de $3~m$ d'arête. Il est le fleuron des technologies d'affichage immersif chez Renault. L'image de chaque face est affichée à l'aide de deux projecteurs 4K haute luminosité. Ces caractéristiques hors-norme permettent à IRIS d'avoir d'excellente notes que ce soit pour le contraste (80), le champ de vision (100), les acuités (98 et 67). Tous les calculs nécessitants la position de l'utilisateur sont fait par rapport à la position usuelle d'utilisation, c'est à dire à $1.30~m$ de la face avant, centré en largeur et avec une hauteur de tête à $1~m$ du sol. ....  \ref{fig:radar_score_iris}

\begin{figure}
	\renewcommand\thefigure{III}
	\centering
	\includegraphics[scale=1]{Figures/RadarScoreIRIS}
	\caption{Diagramme radar des critères du score pour IRIS.}
	\label{fig:radar_score_iris}
\end{figure}

\par P3I \ref{fig:radar_score_p3i}

\begin{figure}
	\renewcommand\thefigure{IV}
	\centering
	\includegraphics[scale=1]{Figures/RadarScoreP3I}
	\caption{Diagramme radar des critères du score pour P3I.}
	\label{fig:radar_score_p3i}
\end{figure}

\par VIVE \ref{fig:radar_score_vive}

\begin{figure}
	\renewcommand\thefigure{V}
	\centering
	\includegraphics[scale=1]{Figures/RadarScoreVIVE}
	\caption{Diagramme radar des critères du score pour le casque VIVE.}
	\label{fig:radar_score_vive}
\end{figure}

\par Hololens \ref{fig:radar_score_hololens}

\begin{figure}
	\renewcommand\thefigure{VI}
	\centering
	\includegraphics[scale=1]{Figures/RadarScoreHololens}
	\caption{Diagramme radar des critères du score pour le casque Hololens.}
	\label{fig:radar_score_hololens}
\end{figure}

\par Conclusion et discussion

\section*{Perspectives d'utilisation}
\par Pour terminer ce manuscrit, on propose de présenter des idées concrètes d'utilisation de notre score de réalisme. Le premier usage étant évidemment celui de l'aide à la conception ou à la mise à jour des systèmes immersifs. Lorsque que l'on doit par exemple améliorer les caractéristiques d'un système de RV, les critères disposant déjà d'une bonne note ne seront pas à traiter en premier. De même, une partie du système ayant atteint le score maximal, ne nécessite à priori pas d'amélioration, à moins que cette dernière ait un effet direct sur un autre critère.

\par On peut ensuite imaginer un usage visant à déterminer la lisibilité dans un simulateur: en pondérant tous les critères liés à cette tâche (acuités, contraste et luminance, fluidité en cas de haute vitesse) et en réglant la pondération des critères moins inutiles à 0. On pourra alors estimer quel moyen immersif est le plus adapté à des expérimentations impliquant de petits détails au loin comme la lecture en amont de panneaux d'autoroute (destination, limitations de vitesse, ...).

\par Enfin, il existe depuis 2016 une initiative permettant de mettre en location ses systèmes immersifs pour un usage quelconque: VR-BNB. Les entreprises (ou laboratoires) créent alors une page décrivant le système qu'elles mettent à disposition: on peut imaginer intégrer sur cette page, ou dans les critères de recherche, le score de réalisme dudit système, en pondération globale ou pondéré dans un cas d'utilisation spécifique. Les futurs utilisateurs auraient alors plus d'indications sur la capacité du système qu'ils envisagent de louer à répondre à leurs besoins expérimentaux. % Date et nom ?