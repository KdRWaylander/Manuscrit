\chapter*{Conclusion}
\addcontentsline{toc}{chapter}{Conclusion}
\markboth{CONCLUSION}{}
\par Dans cette seconde partie expérimentale, nous avons cherché à approfondir notre compréhension de la relation entre la latence et l'expérience utilisateur, dans un contexte d'immersion. Nous avons donc confronté des sujets à la même situation dans différents moyens immersifs, à différents niveaux de latence. Notre approche se voulant objective, nous nous sommes orientés vers des mesures de performance pour tirer des conclusions.

\par Si notre expérimentation confirme certains points de la littérature tels que l'influence de la latence sur la performance, sur le mal du simulateur ou la corrélation entre ce dernier et la présence, elle nous permet également d'apporter nos propres conclusions. D'abord, la variation de la performance semble être non-linéaire et régie par des seuils. Ensuite, ces seuils sont différents suivant les mouvements réalisés: dans la tâche que nous demandions à nos sujets de réaliser, les mouvements verticaux étaient plus petits et plus sensibles que les mouvements horizontaux. Enfin, et contrairement à la performance, le mal du simulateur semble lui évoluer graduellement avec la latence.

\par Ces dernières constatations nous permettent donc de mieux caractériser le critère de latence pour le score de réalisme, sans pour autant le valider complètement: d'une part la limite basse de fonctionnement, associée à la note minimale du score, n'a pas pu être dégagée clairement, et d'autre part, les valeurs de seuils sont spécifiques à la nature des instructions données et devraient être confrontées à d'autres mouvements ou situations de vitesse.