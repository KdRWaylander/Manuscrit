\chapter*{Conclusion}
\addcontentsline{toc}{chapter}{Conclusion}
\markboth{CONCLUSION}{}
\par La première approche de la thèse, développer un <<~modèle de vision~>> traduit pour la Réalité Virtuelle, s'est vite révélée être une impasse. On choisit alors une approche différente: plutôt que de <<~faire~>> du réalisme, il pourrait être intéressant de <<~quantifier~>> le réalisme. A notre connaissance, il existe peu voire très peu de modèles ayant pour but de quantifier objectivement le réalisme physiologique dans la littérature.

\par On propose alors une évaluation de la performance d'un système via un score basé sur le système visuel humain qui dépeindrait, pour un système immersif donné et pour une situation donnée, à quel point le simulateur est efficace pour transmettre le bon niveau d'informations visuelles et, étant dans un système de Réalité Virtuelle, le bon niveau d'indices d'immersion. Ce score est en fait la réunion de douze critères chacun lié à une caractéristique de la vision et/ou de l'immersion visuelle (c'est à dire les critères immersifs dérivés directement du système visuel ou du système en lui-même). Chaque critère est jugé indépendamment puis contribue à la note globale via une pondération. La notation en elle même se fait, principalement, basée sur la littérature. Si on a pu appliquer cette méthode pour la majorité des cas, certains critères ne se prêtent pas à un tel traitement et ont bénéficié d'une notation particulière.

\par Les critères sont répertoriés en deux sections:
	
\begin{itemize}\itemsep12pt
	\item \textbf{Indices de vision:} contraste et luminosité (luminance), images par seconde, nombre de couleurs affichables, champ de vision, acuités monoscopique et stéréoscopique.
	\item \textbf{Indices d'immersion:} latence, champ de regard, stéréoscopie, tracking, uniformité et convergence des caméras.
\end{itemize}
	
\par Après une étude théorique fouillée pour déterminer la plus grande majorité des critères, on a réalisé une expérimentation pour comparer les notes données par notre modèle de score et les notes données par des utilisateur. Cette expérimentation n'avait pas pour but -et ne peut pas- valider notre modèle mais elle nous a permis de montrer la différence qu'il peut exister entre l'appréciation d'un critère noté subjectivement et sa note de performance théorique.

\par Néanmoins, il reste des points à développer. Certains critères ne sont pas aboutis et doivent être complétés par la pratique. C'est le cas par exemple du contraste et de la luminance, ainsi que de la latence. Le traitement expérimental de ces critères sera abordé dans les deux prochaines parties. Enfin, même si on a pu rapidement aborder la pondération intra-critère (champ de vision et champ de regard) ainsi que la légitimité des hypothèses sur lesquelles elle est construite (étude expérimentale), il restera encore tout un travail autour de la pondération inter-critères. Le sujet est extrêmement vaste et complexe car il dépend fortement de ce que l'on fait dans le système immersif.

\par Nous allons maintenant présenter le travail d'expérimentation qui a été réalisé autour de la validation du modèle de Rea en Réalité Virtuelle, pour le critère <<~contraste - luminance~>>.