% CLASSE PRINCIPALE DU DOCUMENT
	\documentclass[a4paper,twoside,11pt,openright,fleqn]{book}
	% options: openright, openany
	
% BIBLIOGRAPHIE (A load avant frenchb/babel sinon warning)
	\usepackage[square,sort,semicolon]{natbib}
	\bibliographystyle{authordate1}
	\usepackage[nottoc]{tocbibind} %Pour que la biblio soit dans le sommaire

% PACKAGES BASIQUES
	\usepackage[english,frenchb]{babel} %Français
	\frenchbsetup{StandardLists = true}	% Remplace le tiret par un point dans les listes
	\usepackage[utf8]{inputenc} %UTF-8 (Accents)
	\usepackage{graphicx} %Ressources graphiques
	\usepackage[T1]{fontenc}
	
% CHAPITRES CUSTOM
	\usepackage[Rejne]{fncychap}
	% Options: Sonny, Lenny, Glenn, Conny, Rejne

% MARGES (Toujours avant fancyhdr)
	\usepackage[top=1in, bottom=1.25in, left=1in, right=1in]{geometry}

% EN-TÊTE ET PIED DE PAGE CUSTOM
	\usepackage{fancyhdr}
	\setlength{\headheight}{15pt}
	\pagestyle{fancy}
	
	% Define fancy styles
	\fancypagestyle{front}{%
		\fancyhf{} % Reset du fancy déjà existant
		\fancyfoot[RO]{\thepage} % Numéro de la page en bas à droite sur les pages paires
		\fancyfoot[LE]{\thepage} % Numéro de la page en bas à gauche sur les pages impaires
		\renewcommand{\headrulewidth}{0pt} % Ligne de tête
		\renewcommand{\footrulewidth}{0.5pt} % Ligne de pied
	}
	
	\fancypagestyle{front_light}{%
		\fancyhf{} % Reset du fancy déjà existant
		\fancyfoot[CO]{\thepage} % Numéro de la page au centre sur les pages paires
		\fancyfoot[CE]{\thepage} % Numéro de la page au centre sur les pages paires
		\renewcommand{\headrulewidth}{0pt} % Ligne de tête
		\renewcommand{\footrulewidth}{0pt} % Ligne de pied
	}
	
	\fancypagestyle{main}{%
		\fancyhf{} % Reset du fancy déjà existant
		\fancyhead[C]{\leftmark} % Nom du chapitre en haut au centre	
		\fancyfoot[RO]{\thepage} % Numéro de la page en bas à droite sur les pages paires
		\fancyfoot[LE]{\thepage} % Numéro de la page en bas à gauche sur les pages impaires
		\renewcommand{\headrulewidth}{0.5pt} % Ligne de tête
		\renewcommand{\footrulewidth}{0.5pt} % Ligne de pied
	}
	
	\fancypagestyle{star}{%
		\fancyhf{} % Reset du fancy déjà existant
		\fancyfoot[RO]{\thepage} % Numéro de la page en bas à droite sur les pages paires
		\fancyfoot[LE]{\thepage} % Numéro de la page en bas à gauche sur les pages impaires
		\renewcommand{\headrulewidth}{0.5pt} % Ligne de tête
		\renewcommand{\footrulewidth}{0.5pt} % Ligne de pied
	}
	
	\fancypagestyle{back}{%
		\fancyhf{} % Reset du fancy déjà existant
		\fancyfoot[RO]{\thepage} % Numéro de la page en bas à droite sur les pages paires
		\fancyfoot[LE]{\thepage} % Numéro de la page en bas à gauche sur les pages impaires
		\renewcommand{\headrulewidth}{0pt} % Ligne de tête
		\renewcommand{\footrulewidth}{0.5pt} % Ligne de pied
	}
	
	\fancypagestyle{part}{%
		\fancyhf{} % Reset du fancy déjà existant
		\fancyfoot[RO]{\thepage} % Numéro de la page en bas à droite sur les pages paires
		\fancyfoot[LE]{\thepage} % Numéro de la page en bas à gauche sur les pages impaires
		\renewcommand{\headrulewidth}{0pt} % Ligne de tête forcée invisible
		\renewcommand{\footrulewidth}{0pt} % Ligne de pied forcée invisible
	}

	% Redefine the plain page style
	\fancypagestyle{plain}{%
  		\fancyhf{}%
  		\fancyfoot[RO]{\thepage} % Numéro de la page en bas à droite sur les pages paires
		\fancyfoot[LE]{\thepage} % Numéro de la page en bas à gauche sur les pages impaires
		\renewcommand{\headrulewidth}{0pt} % Ligne de tête forcée invisible
		\renewcommand{\footrulewidth}{0.5pt} % Ligne de pied
  	}
  	
  	% Empty style on PART pages
  	\makeatletter
    \renewcommand\part{%
       \if@openright
         \cleardoublepage
       \else
         \clearpage
       \fi
       \thispagestyle{empty}%
       \if@twocolumn
         \onecolumn
         \@tempswatrue
       \else
         \@tempswafalse
       \fi
       \null\vfil
       \secdef\@part\@spart}
    \makeatother 
	
% TAILLE DES TITRES
	\usepackage{titlesec}
	\titleformat*{\section}{\LARGE\bfseries}
	\titleformat*{\subsection}{\Large\bfseries}
	\titleformat*{\subsubsection}{\large\bfseries}
	\titleformat*{\paragraph}{\large\bfseries}
	\titleformat*{\subparagraph}{\large\bfseries}

% ALPHANUM FOR APPENDIXES
	\usepackage{alphalph}
	
% PARAGRAPHE SPACING
	\setlength{\parskip}{\baselineskip}
	\setlength{\parindent}{0pt}
	
% PAGE DE GARDE
	\usepackage{lmodern}
	\usepackage{ae,aecompl}
	\usepackage{eso-pic}	% Nécessaire pour mettre des images en arrière plan
	\usepackage{array}		% Permet d'écrite 'THESE' de haut en bas
	\usepackage[letterspace=-60]{microtype}
	\renewcommand{\rmdefault}{pnc} 
	\renewcommand{\sfdefault}{pag} 
	\renewcommand{\ttdefault}{pcr}
	
% ECRITURE MATHEMATIQUE
	\usepackage{amsmath}
	
% CONVERTION .eps VERS .pdf
	\usepackage{epstopdf}
	
% TABLEAUX
	{\renewcommand{\arraystretch}{1.5} %Taille des interlignes
	\usepackage{multirow}
	
% LISSAGE DE LA POLICE (Problème de hauteurs de certaines lettres)
	\renewcommand{\rmdefault}{pnc}
	\renewcommand{\sfdefault}{pag} 
	\renewcommand{\ttdefault}{pcr}
	
% LETTRINES
	\usepackage{lettrine}
	
% LIENS
	\usepackage{hyperref} %Liens dynamiques (Sommaire)
% Style dans le sommaire
	\hypersetup{
	    colorlinks,
	    citecolor=black,
	    filecolor=black,
	    linkcolor=black,
	    urlcolor=black
	}

% NUMEROTATION DES FOOTNOTES PAR PAGE
	\usepackage{perpage}
	\MakePerPage{footnote}
	
% INCLURE DES PDFs
	\usepackage{pdfpages}
	
% TABLE DES MATIERES PAR CHAPITRE
	%\usepackage{minitoc}
	%\setcounter{minitocdepth}{2}
	%\setlenght{\mtcindent}{24pt}
	%\renewcommand{\mtcfont}{\small\rm}
	%\renewcommand{\mtcSfont}{\small\bf}

% INDEX
	%\usepackage{imakeidx}
	%\makeindex[columns=2, title=Index, intoc]
	
% NOMENCLATURE
	%\usepackage{nomencl}
	%\makenomenclature[title=Nomemclature, intoc]
	
% AJOUTER UNE PAGE BLANCHE
	\newcommand{\blankpage}{
	\newpage
	\mbox{}
	\newpage	
	}
	
% RESET CHAPTER COUNTER FOR PARTS
	%\makeatletter
	%\@addtoreset{chapter}{part}
	%\makeatother
	
% NUMEROTATION DES FIGURES/TABLES/EQUATIONS INDEP DU CHAPITRE
	\usepackage{chngcntr}
 
	\counterwithout{figure}{chapter}
	\counterwithout{table}{chapter}
	\counterwithout{equation}{chapter}

% BIBLIO AS PART
	\makeatletter
	\renewenvironment{thebibliography}[1]
     	{\part*{\bibname}
     	 \@mkboth{\MakeUppercase\bibname}{\MakeUppercase\bibname}%
    	  \list{\@biblabel{\@arabic\c@enumiv}}%
     	      {\settowidth\labelwidth{\@biblabel{#1}}%
        	    \leftmargin\labelwidth
            	\advance\leftmargin\labelsep
            	\@openbib@code
            	\usecounter{enumiv}%
            	\let\p@enumiv\@empty
            	\renewcommand\theenumiv{\@arabic\c@enumiv}}%
      	\sloppy
      	\clubpenalty4000
      	\@clubpenalty \clubpenalty
      	\widowpenalty4000%
      	\sfcode`\.\@m}
     	{\def\@noitemerr
       	{\@latex@warning{Empty `thebibliography' environment}}%
      	\endlist}
	\makeatother
	
% RENAME DES INSERTIONS AUTOMATIQUES (Laisser à la fin du fichier Style)
	\renewcommand{\listfigurename}{Liste des Figures}		%Figures
	\renewcommand{\listtablename}{Liste des Tables}			%Tables
	\renewcommand{\contentsname}{Sommaire}					%Sommaire
	\renewcommand{\bibname}{Références}						%Refs