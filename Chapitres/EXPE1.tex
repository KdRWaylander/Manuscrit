%\part{Partie expérimentale: contraste et luminance}
\part{Expérimentation: confrontation d'un modèle de performance visuelle pour le critère de contraste - luminance}

\chapter*{Introduction}
\addcontentsline{toc}{chapter}{Introduction}
\markboth{INTRODUCTION}{}
\par Dans le chapitre précédent, on a décrit deux approches possibles pour le critère de contraste et de luminance: les fonctions de sensibilité au contraste et la performance visuelle relative. On s'intéresse ici au concept de performance visuelle. On en a présenté les deux principales modélisations: le modèle de Blackwell et de la CIE \citep{blackwell_ieri:_1971}, puissant mais hermétique, et le modèle de Rea et ses évolutions \citep{rea_toward_1986}. On a également pointé le fait que ces modèles, bien qu'intéressants, n'étaient pas à l'origine conçus pour la réalité virtuelle et nécessitaient donc une vérification expérimentale.

\par Il a été nécessaire de choisir entre les deux modèles, celui de Rea et celui de la CIE et Blackwell. Le modèle de Blackwell semble plus complet avec une portée d'action comprise entre $1$ et $10000~cd/m^2$ et un calcul basé sur trois processus de vision, décrits comme principaux, impliqués dans la reconnaissance des détails de la tâche à effectuer, dans le maintien des yeux en position fixe pendant les périodes de fixation, et enfin, dans la réalisation de mouvements rapides des yeux (les saccades). Néanmoins, le modèle de la CIE est une boite noire générée à partir de la mise en commun des travaux d'un certain nombre de chercheurs. Il n'existe pas d'expérimentation détaillée qui puisse être refaite, et à fortiori, encore moins en réalité virtuelle.

\par De l'autre côté, la modèle de Rea est plus limité en portée (entre $12$ et $169~cd/m^2$) mais présente en détail tout le protocole qui a été mis en œuvre pour développer les équations de performance visuelle. De plus, l'intervalle de fonctionnement du modèle correspond relativement bien aux luminances atteignables dans un simulateur et n'est donc pas très contraignant.

\par L'objectif de cette partie est donc de vérifier par l'expérimentation, dans un simulateur, les prédictions de performance du modèle de Rea. Pour ce faire, on a transposé en Réalité Virtuelle l'une des expérimentations mise en place par Rea et Ouellette pour déterminer leur modèle.