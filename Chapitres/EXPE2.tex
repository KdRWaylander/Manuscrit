\part{Expérimentation: latence}

\chapter*{Introduction}
\addcontentsline{toc}{chapter}{Introduction}
\markboth{INTRODUCTION}{}
\par Dans cette deuxième partie expérimentale on s'intéresse cette fois à la latence. Il est important de définir d'entrée ce que l'on entend par <<~latence~>> car il peut co-exister de nombreuses définitions \citep{papadakis_system_2011, hale_handbook_2015, watson_effects_1998}. Nous utiliserons ici la plus commune et la plus générale: la latence <<~end-to-end~>>, c'est à dire la latence globale de <<~bout en bout~>>. Celle-ci correspond au temps (en générale en millisecondes) qui s'écoule entre un mouvement réel dans un environnement immersif (que ce soit de la tête, de la main ou de tout autre objet tant que celui-ci est tracké) et l'affichage du mouvement en question sur le/s écran/s de l'environnement immersif. Évidemment, dans notre cas, c'est la latence liée au mouvement de la tête que nous allons évaluer.

\par L'objectif est ici d'alimenter le critère de latence de notre modèle en informations sur la répercussion que peut avoir la latence sur le système visuel directement mais également de manière plus globale sur l'immersion, la présence et l'expérience utilisateur. Ces informations pourraient nous permettre d'établir des guidelines pour améliorer autant que possible l'expérience utilisateur dans les systèmes immersifs.

\par A la différence de la partie précédente, nous ne chercherons pas à valider un modèle déjà existant. Il existe des valeurs de seuil pour la latence (que nous décrirons dans le chapitre suivant) bien que difficiles à atteindre dans le cadre d'expérimentation écologiques (le nombre de triangles, la complexité du système ou le nombre d'intermédiaires sont autant de paramètres dégradant la réactivité du système). Cependant, de manière analogue à la partie précédente, l'axe de la performance est privilégié pour observer le plus objectivement possible l'influence de la latence sur le système visuel humain au sens large.

\par Cette partie décrit donc l'expérimentation qui a été menée, sur deux systèmes immersifs différents, dans un contexte de tâche similaire à ce que l'on pourrait exécuter dans la vie de tous les jours, avec des conditions de latence variables.