\chapter*{Introduction générale}
\addcontentsline{toc}{part}{Introduction générale}
\markboth{INTRODUCTION GÉNÉRALE}{}

\lettrine[lines=3]{L}{es} travaux de la thèse <<~Immersion visuelle hyper-réaliste et multi-sensorielle 3D~>>, dont les résultats sont présentés ici, ont été réalisés dans le cadre d’un contrat CIFRE établi entre les Arts et Métiers et Renault, au sein du laboratoire commun LIV. Ils sont le fruit d'un travail de trois ans, réalisé en grande partie au Technocentre Renault à Guyancourt, sous l'encadrement de Stéphane Régnier, expert maquette virtuelle pour l'ingénierie et le design ; Fédéric Mérienne, professeur aux Arts et Métiers et directeur de l'Institut Image ; et Andras Kemeny, expert-leader simulation immersive et réalité virtuelle.

\par Renault, fabriquant français de voitures, investit de plus en plus de moyens dans la réalité virtuelle, et même depuis peu, dans la réalité augmentée. Ces technologies viennent compléter les outils de développement des nouveaux véhicules. Ils permettent de tester, d'afficher ou de rouler sans avoir besoin de passer par des prototypes physiques et dans les conditions que l'on souhaite (brouillard, nuit, pluie, ...). Ces procédés permettent de nombreux gains de temps et d'argent, composantes vitales pour l'entreprise.

\par L'Institut Image, institut de recherche rattaché au campus Arts et Métiers de Cluny, a quant à lui pour objectifs le développement d'outils et de méthodes pour l'immersion virtuelle, au service de l'ingénieur. Ses missions sont celles de la formation, de la recherche et de l'innovation. Dans le but de collaborer à des projets industriels concrets, Renault et les Arts et Métiers, via l'Institut Image, crèent en 2011 un laboratoire commun: le LIV (Laboratoire d'Immersion Virtuelle). C'est dans le cadre de ce partenariat qu'a eu lieu la présente thèse.

\par Notre étude est née de la volonté de Renault d'offrir à ses équipes des moyens sûrs, permettant la prise de décisions parfois stratégiques, sans craindre un biais causé par la technique. Les outils de travail doivent proposer des situations les plus fidèles à la réalité possible, c'est à dire <<~réalistes~>>. L'objectif de la thèse est donc de formuler ce qu'est le réalisme et de quelle manière il peut être atteint ou réalisé. Si le sujet peut sembler un peu large initialement, nous concentrons nos efforts sur la partie matérielle des moyens immersifs, avec en priorité ceux disponibles pour le laboratoire commun. Renault s'associe donc à un partenaire universitaire, les Arts et Métiers. C'est ici qu'entre en jeu la spécificité d'un contrat de type CIFRE: les travaux ont une vocation universitaire, c'est à dire qu'ils sont appelés à être communiqués, publiés; mais ils possèdent également une réelle vocation applicative, avec des attendus concrets de la part de l'entreprise, qui peuvent être confidentiels. Il est alors dévolu au doctorant de trouver le juste équilibre entre les attentes des deux parties.

\par Le manuscrit est présenté en cinq grandes parties: deux parties à dominante théorique, une partie modélisation puis deux parties expérimentales.

\par La première partie présente la synthèse de l'état de l'Art qui a été effectué afin de comprendre en profondeur les différent(e)s mécanismes et modélisations qui prennent part au fonctionnement de l'œil, de la vision, et des premières phases de la perception visuelle.

\par La deuxième partie s'intéresse spécifiquement aux modèles de vision et retrace la première approche qui a été envisagée, puis abandonnée, pour la thèse.

\par Ensuite, la troisième partie présente la proposition de modèle au cœur de nos travaux: ses objectifs, sa mise en place et ses différentes composantes.

\par D'abord très théorique, la thèse se dirige ensuite vers deux segments plus expérimentaux. Dans la quatrième partie, on décrit la traduction et l'implémentation d'un modèle de performance visuelle, en Réalité Virtuelle.

\par Enfin, dans la cinquième et dernière partie, on s'intéresse à la latence, ses effets sur la performance et l'expérience utilisateur, en comparant deux moyens immersifs.