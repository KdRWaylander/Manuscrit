\chapter*{Conclusion}
\addcontentsline{toc}{chapter}{Conclusion}
\markboth{CONCLUSION}{}
\par L'objectif de cette première expérimentation était d'alimenter le critère <<~contraste et luminance~>> de notre modèle de score pour la Réalité Virtuelle. Le postulat de base était qu'il existe des modèles qui définissent une performance, entièrement basée sur le système visuel, et qui prennent notamment en paramètre d'entrée le contraste et la luminance. Ces modèles n'ont cependant pas été établis pour la Réalité Virtuelle, ni même, plus simplement, pour de grands écrans. On cherche alors à démontrer la transposabilité d'un de ces modèles (Rea et Ouellette) dans notre domaine afin de pouvoir l'utiliser directement dans notre score.

\par La démarche était donc de reproduire le protocole expérimental ayant servi à établir le modèle initial, mais en le transposant  dans un système immersif, et de quantifier la bonne corrélation entre les valeurs prévues par le modèle et celles mesurées sur des sujets. Les résultats montrent un comportement qui diverge pour les basses conditions de contraste et de luminosité, mais qui converge pour les hautes conditions. Il existe néanmoins une corrélation statistique entre les données mesurées et les données théoriques. La taille de l'arrière plan, et par extension la luminance produite, semblent être la raison de cette différence: pour les basses conditions de contraste, plus le fond sur lequel est affiché le stimulus est grand, plus la sensibilité de l'observateur est bonne. Cet effet disparait quand les conditions lumineuses s'améliorent, ce qui explique la convergence des résultats d'un côté (hautes conditions) et le rehaussement de la performance de l'autre (basses conditions).

\par On ne peut donc pas utiliser directement le modèle de Rea et Ouellette pour déterminer le score du critère de contraste et de luminance. Il serait alors nécessaire de proposer une modification du modèle ou d'établir le notre.

\par Néanmoins, avec l'objectif de couvrir l'intégralité des critères de notre modèle, ainsi que sa pondération, on continue nos travaux avec une expérimentation sur la latence. On consacre la prochaine partie à la description et l'analyse des résultats de cette dernière.