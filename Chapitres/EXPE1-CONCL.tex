\chapter*{Conclusion}
\addcontentsline{toc}{chapter}{Conclusion}
\par L'objectif de cette première expérimentation était d'alimenter la brique <<~contraste luminance~>> de notre modèle de score pour la Réalité Virtuelle. Le postulat de base était qu'il existe des modèles qui définissent une performance, entièrement basée sur le système visuelle, et qui prennent notamment en paramètre d'entrée le contraste et la luminance, mais ces modèles n'ont pas été fait pour la Réalité Virtuelle, ni même pour des écrans quelconques. On cherche alors a démontrer la transposabilité d'un de ces modèles dans notre domaine afin de pouvoir l'utiliser directement dans notre score.

\par La démarche était donc de reproduire le protocole expérimental ayant servi à établir le modèle initial, mais en le transposant  dans un simulateur, et de quantifier la bonne corrélation entre les valeurs prévues par le modèle et celles mesures sur des sujets. Les résultats sont de ce point de vue là décevant car ils montrent un comportement radicalement différent de ce qui était prévu. Bien que l'on ait quelques hypothèses sur l'origine de cette différence de comportement, cette dernière reste encore difficile à expliquer pleinement.

\par On peut établir un certain nombre de choses de cette expérimentation. Premièrement, le modèle de Rea, et plus particulièrement la variante développée par Rea et Ouellette avec des temps de réaction, ne semble pas fonctionner pour la Réalité Virtuelle. Ensuite, et par extension, on ne peut pas utiliser directement le modèle de Rea comme on le souhaitait pour établir le critère de <<~contraste/luminance~>> de notre modèle de score. Cela n'implique pas néanmoins l'impossibilité de corréler notre critère avec un autre modèle de performance visuelle.

\par Se pose alors la question de la continuité à donner aux résultats de cette expérimentation. On peut choisir de continuer à approfondir la performance visuelle, en créant par exemple notre propre modèle, mais on choisit de s'intéresser à une autre brique élémentaire de notre modèle de score: la latence. Ce changement de sujet permet de travailler plusieurs aspects du modèle et de ne pas s'enfermer dans une seule branche. Le chapitre suivant concerne donc la partie expérimentale autour du traitement du critère suivant, la latence.