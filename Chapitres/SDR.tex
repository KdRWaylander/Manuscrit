%\part{Deuxième approche: score de réalisme}
\part{Score de réalisme visuel en Réalité Virtuelle}
	\chapter*{Introduction}
	\addcontentsline{toc}{chapter}{Introduction}
	\markboth{INTRODUCTION}{}
	\par Pour rappel, on avait posé dans la définition du réalisme (\textit{cf.} Chapitre \ref{part:edla}.\ref{chap:def_realisme}) le cadre d'étude suivant:
	\begin{quote}
		<<~Pour le cas de cette thèse, la définition du réalisme qui aura été retenue -et qui sera sous-entendue quand on utilisera le mot \textit{réalisme} seul- est celle de la proximité physiologique avec le système visuel humain. [...] Le réalisme physiologique doit, par construction et par définition, se baser sur le système visuel humain.~>>
	\end{quote}
	La première approche, qui était de développer un <<~modèle de vision~>> traduit pour la Réalité Virtuelle et les caméras qui extraient les images des scènes 3D, s'est vite révélée être une impasse ou en tout cas une voie offrant peu de débouchés tant ces modèles étaient déjà largement décrits dans la littérature.
	
	\par On choisit alors une approche différente: plutôt que de \textit{faire} du réalisme, il pourrait être intéressant de \textit{quantifier} le réalisme qu'un-e système/application produit. En effet, si on entend souvent parler de réalisme, son appréciation est souvent binaire: soit un système ou une application est réaliste, soit il/elle ne l'est pas. De plus, il est difficile de pouvoir dire concrètement ce qui rend un système moins réaliste qu'un autre. Pour ce faire, il faudrait pouvoir être capable de mettre des valeurs et des critères objectifs à remplir. On aurait alors un moyen efficace et juste de juger du réalisme d'un système et/ou de comparer plusieurs éléments entre eux.
	
	\par Outre sa dimension théorique, on anticipe deux usages potentiels à une telle approche: elle pourrait permettre de choisir un système pour un cas d'usage donné. En fonction des critères dont on a besoin principalement, on pourra se référer aux notes pour faire un choix. De même, le système de notation permettrait de comparer les dispositifs immersifs et ainsi de les positionner entre eux.