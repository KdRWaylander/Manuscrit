\chapter*{Résumé}
\par La thèse <<~Immersion visuelle réaliste: proposition d'un modèle d'évaluation~>> a été réalisée dans le cadre d’un contrat CIFRE établi entre les Arts et Métiers d’une part et Renault SAS d’autre part. Elle propose un modèle de score permettant d’évaluer objectivement de la capacité d’un système d’affichage immersif à reproduire le bon niveau de stimulation sensorielle pour un utilisateur, par rapport à ce qu’il recevrait dans la réalité et à la modélisation du système visuel humain.

\par Dans un premier temps nous nous sommes intéressé à poser les bases du modèle : celui-ci est composé de douze critères, répartis équitablement en une somme d’indices de vision et d’indices d’immersion. Chaque critère se voit attribuer, dans la mesure du possible, une note de 0 à 100. La note de 0 représente l’incapacité du système visuel à percevoir ou à utiliser les informations visuelles, tandis que la note de 100 incarne la capacité maximale. Une note de 80 est également assignée pour la performance standard. Chaque critère se voit assigner une pondération en fonction de la tâche réalisée en environnement virtuel.

\par Nous réalisons dans un second temps une série d’expérimentations afin de compléter les informations disponibles dans la littérature, pour l’établissement des critères. On s’intéresse plus particulièrement au contraste et à la latence. La première expérimentation consiste en la transposition et la validation d’un modèle de performance visuelle en Réalité Virtuelle. Dans la seconde, on compare les effets de la latence sur une tâche donnée, entre un simulateur de type CAVE et un casque de Réalité Virtuelle.

\par La thèse propose un certain nombre de résultats tant théoriques et méthodologiques qu’expérimentaux. En condition d’immersion virtuelle, les critères de contraste et de luminance sont importants pour la perception visuelle. Les expériences réalisées montrent qu’un modèle adapté aux conditions d’immersion virtuelle est nécessaire. Par ailleurs, on montre que des seuils d’influence de la latence sur la performance semblent exister. On vérifie également la pertinence de corrélations entre performance, présence et mal du simulateur, en fonction de la latence et du système immersif.


\chapter*{Abstract}
\par The thesis "Realistic visual immersion: proposal of an evaluation model" was carried out within the framework of a CIFRE contract established between Arts et Métiers on the one hand and Renault SAS on the other. It proposes a score model to objectively evaluate the ability of an immersive display system to reproduce the right level of sensory stimulation for a user, compared to what he would receive in reality and to the modeling of the human visual system.

\par First, we were interested in laying the foundations of the model: it is composed of twelve criteria, equitably divided into a sum of vision indices and immersion indices. Each criterion is given, as far as possible, a score from 0 to 100. A score of 0 represents the visual system's inability to perceive or use visual information, while a score of 100 represents maximum capacity. A score of 80 is also assigned for standard performance. Each criterion is assigned a weight according to the task performed in the virtual environment.

\par We then carry out a series of experiments to complete the information available in the literature, to establish the criteria. Contrast and latency are of particular interest. The first experiment consists in the transposition and validation of a visual performance model in Virtual Reality. In the second, we compare the effects of latency on a given task, between a CAVE simulator and a HMD.

\par The thesis proposes a number of theoretical, methodological and experimental results. In virtual immersion conditions, contrast and luminance criteria are important for visual perception. Experiments show that a model adapted to virtual immersion conditions is necessary. Furthermore, we show that thresholds of latency influence on performance seem to exist. The relevance of correlations between simulator performance, presence and cyber-sickness is also checked, depending on the latency and the immersive system.

\chapter*{Remerciements}
\par Premièrement, je voudrais remercier Stéphane, mon tuteur, sans qui tout ce travail n'aurait pas été possible. Son expérience du terrain, ses idées, ses remarques et propositions sont à la base du reste. Je remercie également Andras et Frédéric qui ont grandement contribué à l'aboutissement de mes travaux par leurs conseils, remarques, et participations. Ensuite, j'aimerais remercier l'ensemble de mon jury: Mme Pascaline Neveu, MM. Philippe Fuchs, Ronan Querrec, Daniel Mestre et David Alleysson pour avoir participé à leur manière à la conclusion de mes travaux, par leurs remarques orales et écrites, ainsi que toutes les discussions que nous avons pu avoir lors de la soutenance.

\par Je remercie également tous les collègues que j'ai pu côtoyer chez Renault pendant la durée de ma thèse, qu'ils soient embauchés, stagiaires ou doctorants. En premier lieu mes remerciements vont à Mhappy, avec qui j'ai passé le plus clair de mon temps; à Fen, Quentin et Paul, qui ont été pour moi des <<~grands frères~>> de thèse, mais également à Carolina, Guillaume, Bérénice, Lucie, Valentin, Vincent, Thibault, Javier. Sans tous les nommer, je remercie le reste des collègues du service ainsi que tous les prestas.

\par Au delà du cercle de la thèse, je remercie aussi les amis qui me sont proches, que ce soient ceux de Lyon (Antoine, Aymeric, Caroline, Marie ou Guillaume) ou ceux de Cluny (Exo, Mugen, Arémuse, Chouchou, Gamma, Celli, Tadash, Zaru et Alex, Sisi et Nico, Chocol, Alto, Didi, Faf, Macharius et tous les autres).

\par J'ai une pensée également pour les médecins, soignants, professeurs et encadrants qui m'ont soutenu et qui ont permis l'aboutissement de mon parcours: le Pr Bejui, le Dr Guyard, le Dr Marec-Berard, M. Garden, Mme Ohler, M. Thomas et beaucoup d'autres encore. 

\par Je remercie profondément mes parents, mon frère, mes sœurs et Fabien qui constituent la garde rapprochée depuis toujours et qui sont les premiers responsables de ce que je suis aujourd'hui. Je remercie également mes grand-parents, mon parrain et ma marraine, ainsi que tous mes oncles, tantes, cousins et cousines. Enfin, il me reste à remercier celle qui partage ma vie de tous les jours, merci.
