% CLASSE PRINCIPALE DU DOCUMENT
	\documentclass[a4paper,twoside,11pt,openany,fleqn]{book}
	% option openright pour forcer les débuts de chapitres sur page impaire
	% sinon, openany
	
% BIBLIOGRAPHIE (A load avant frenchb/babel sinon warning)
	\usepackage[square,sort,semicolon]{natbib}
%	\usepackage[sectionbib]{chapterbib}
	\bibliographystyle{authordate1}
	\usepackage[nottoc]{tocbibind} %Pour que la biblio soit dans le sommaire

% PACKAGES BASIQUES
	\usepackage[english,frenchb]{babel} %Français
	\frenchbsetup{StandardLists = true}	% Remplace le tiret par un point dans les listes
	\usepackage[utf8]{inputenc} %UTF-8 (Accents)
	\usepackage{graphicx} %Ressources graphiques
	\usepackage[T1]{fontenc}
	
% CHAPITRES CUSTOM
	\usepackage[Rejne]{fncychap}
	% Options: Sonny, Lenny, Glenn, Conny, Rejne
	
% EN-TÊTE ET PIED DE PAGE CUSTOM
	\usepackage{fancyhdr}
	\usepackage{etoolbox}
	\setlength{\headwidth}{450pt}
	\pagestyle{fancy}
	
	% Define fancy styles
	\fancypagestyle{front}{%
		\fancyhf{} % Reset du fancy déjà existant
		\fancyfoot[RO]{\thepage} % Numéro de la page en bas à droite sur les pages paires
		\fancyfoot[LE]{\thepage} % Numéro de la page en bas à gauche sur les pages impaires
		\renewcommand{\headrulewidth}{0pt} % Ligne de tête
		\renewcommand{\footrulewidth}{0.5pt} % Ligne de pied
	}
	
	\fancypagestyle{main}{%
		\fancyhf{} % Reset du fancy déjà existant
		\fancyhead[C]{\large \thepart: \normalsize \leftmark} % Nom du chapitre en haut au centre	
		\fancyfoot[RO]{\thepage} % Numéro de la page en bas à droite sur les pages paires
		\fancyfoot[LE]{\thepage} % Numéro de la page en bas à gauche sur les pages impaires
		\renewcommand{\headrulewidth}{0.5pt} % Ligne de tête
		\renewcommand{\footrulewidth}{0.5pt} % Ligne de pied
	}
	
	\fancypagestyle{back}{%
		\fancyhf{} % Reset du fancy déjà existant
		\fancyfoot[RO]{\thepage} % Numéro de la page en bas à droite sur les pages paires
		\fancyfoot[LE]{\thepage} % Numéro de la page en bas à gauche sur les pages impaires
		\renewcommand{\headrulewidth}{0pt} % Ligne de tête
		\renewcommand{\footrulewidth}{0.5pt} % Ligne de pied
	}

	% Redefine the plain page style
	\fancypagestyle{plain}{%
  		\fancyhf{}%
  		\fancyfoot[RO]{\thepage} % Numéro de la page en bas à droite sur les pages paires
		\fancyfoot[LE]{\thepage} % Numéro de la page en bas à gauche sur les pages impaires
		\renewcommand{\headrulewidth}{0pt} % Ligne de tête forcée invisible
		\renewcommand{\footrulewidth}{0.5pt} % Ligne de pied
  	}
	
% TAILLE DES TITRES
	\usepackage{titlesec}
	\titleformat*{\section}{\LARGE\bfseries}
	\titleformat*{\subsection}{\Large\bfseries}
	\titleformat*{\subsubsection}{\large\bfseries}
	\titleformat*{\paragraph}{\large\bfseries}
	\titleformat*{\subparagraph}{\large\bfseries}

% MARGES
	\usepackage[top=1in, bottom=1.25in, left=1in, right=1in]{geometry}
	
% PARAGRAPHE SPACING
	\setlength{\parskip}{\baselineskip}
	\setlength{\parindent}{0pt}
	
% ECRITURE MATHEMATIQUE
	\usepackage{amsmath}
	
% CONVERTION .eps VERS .pdf
	\usepackage{epstopdf}
	
% TABLEAUX
	{\renewcommand{\arraystretch}{1.5} %Taille des interlignes
	
% LISSAGE DE LA POLICE (Problème de hauteurs de certaines lettres)
	\renewcommand{\rmdefault}{pnc}
	\renewcommand{\sfdefault}{pag} 
	\renewcommand{\ttdefault}{pcr}
	
% LETTRINES
	\usepackage{lettrine}
	
% LIENS
	\usepackage{hyperref} %Liens dynamiques (Sommaire)
% Style dans le sommaire
	\hypersetup{
	    colorlinks,
	    citecolor=black,
	    filecolor=black,
	    linkcolor=black,
	    urlcolor=black
	}

% NUMEROTATION DES FOOTNOTES PAR PAGE
	\usepackage{perpage}
	\MakePerPage{footnote}
	
% INCLURE DES PDFs
	\usepackage{pdfpages}
	
% TABLE DES MATIERES PAR CHAPITRE
	%\usepackage{minitoc}
	%\setcounter{minitocdepth}{2}
	%\setlenght{\mtcindent}{24pt}
	%\renewcommand{\mtcfont}{\small\rm}
	%\renewcommand{\mtcSfont}{\small\bf}

% INDEX
	%\usepackage{imakeidx}
	%\makeindex[columns=2, title=Index, intoc]
	
% NOMENCLATURE
	%\usepackage{nomencl}
	%\makenomenclature[title=Nomemclature, intoc]
	
% STYLE SPECIFIQUE
	% Reset chapter counter for parts
	\makeatletter
	\@addtoreset{chapter}{part}
	\makeatother 

	% Biblio as PART
	\makeatletter
	\renewenvironment{thebibliography}[1]
     	{\part*{\bibname}
     	 \@mkboth{\MakeUppercase\bibname}{\MakeUppercase\bibname}%
    	  \list{\@biblabel{\@arabic\c@enumiv}}%
     	      {\settowidth\labelwidth{\@biblabel{#1}}%
        	    \leftmargin\labelwidth
            	\advance\leftmargin\labelsep
            	\@openbib@code
            	\usecounter{enumiv}%
            	\let\p@enumiv\@empty
            	\renewcommand\theenumiv{\@arabic\c@enumiv}}%
      	\sloppy
      	\clubpenalty4000
      	\@clubpenalty \clubpenalty
      	\widowpenalty4000%
      	\sfcode`\.\@m}
     	{\def\@noitemerr
       	{\@latex@warning{Empty `thebibliography' environment}}%
      	\endlist}
	\makeatother
	
% RENAME DES INSERTIONS AUTOMATIQUES (Laisser à la fin du fichier Style)
	\renewcommand{\listfigurename}{Liste des Figures}		%Figures
	\renewcommand{\listtablename}{Liste des Tables}			%Tables
	\renewcommand{\contentsname}{Sommaire}					%Sommaire
	\renewcommand{\bibname}{Références}						%Refs